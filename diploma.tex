\documentclass[12pt,a4paper]{report}

\usepackage[T2A]{fontenc}
\usepackage[russian]{babel}
\usepackage[utf8]{inputenc}
\usepackage{amsmath}
\usepackage{amssymb}
\usepackage{mathtools}
\usepackage{caption}
\usepackage[caption2]{ccaption}
\usepackage{indentfirst}
\usepackage{setspace}
\usepackage{etoolbox}
\usepackage{todonotes}

\renewcommand{\contentsname}{Содержание}
\renewcommand{\bibname}{Список литературы}
\renewcommand{\figurename}{Рис.}
\renewcommand{\tablename}{Таблица}
\renewcommand{\abstractname}{Аннотация}
\renewcommand{\partname}{Часть}

\renewcommand{\bottomfraction}{0.5}
\renewcommand{\floatpagefraction}{0.4}
\renewcommand{\textfloatsep}{0.5cm}
\renewcommand{\intextsep}{0.6cm}
\renewcommand{\floatsep}{0.3cm}

\hoffset -1.6cm
\textwidth  16.5cm
\textheight 24cm

% for todos
\newcommand\note[1]{\textcolor{red}{(#1)}}
\newcommand\todonote[1]{\note{TODO: #1}}

\begin{document}

% Титульный лист
\begin{titlepage}
\par
\vspace*{-4cm}
\begin{center}
{\large

Санкт-Петербургский Государственный Политехнический Университет\\
Институт прикладной математики и механики\\
Кафедра прикладной математики\\

\vspace*{0.5cm}

\begin{flushright}
Диссертация допущена к защите\\
Зав. кафедрой\ \ \ \ \ \ \ \ \ \ \ \ \ \ \ \ \ \ \ \ \ \ \ \ \ \ \ \\
\underline{ \ \ \ \ \ \ \ \ \ \ \ \ \ \ \ \ \ \ \ \ \ \ \ \ } В.Е.Клавдиев\\
"\underline{ \ \ }"\underline{ \ \ \ \ \ \ \ \ \ \ \ \ \ \ \ \ \ \ \ \ \ \ \ \ \ \ \ \ \ \ \ \ \ \ \ \ \ \ }
\end{flushright}

\vspace*{2.0cm}

{\Large
  \textbf{
    ДИССЕРТАЦИЯ\\
    на соискание степени МАГИСТРА\\
  }
}
\vspace*{1cm}
\textbf{
  Тема: \emph{метод ранжирования разнородных результатов поиска}\\
}

}

\vspace*{1.5cm}

\begin{flushleft}
Направление: 010400 - Прикладная математика и информатика\\
Магистерская программа: системное программирование\\
\end{flushleft}

\vspace*{1cm}

Выполнил студент гр. 63601/2 \ \ \ \ \ \ \ \ \ \ \ \ \ \ \ \ \ \ \ \ \ \ \ \ \ \ \ \ \ \ \ \ \ \ \ \ \ \underline{ \ \ \ \ \ \ \ \ \ \ \ \ \ \ \ \ \ \ \ \ \ \ \ } Толмачев А.С.\\
\vspace*{0.3cm}
Руководитель \ \ \ \ \ \ \ \ \ \ \ \ \ \ \ \ \ \ \ \ \ \ \ \ \ \ \ \ \ \ \ \ \ \ \underline{ \ \ \ \ \ \ \ \ \ \ \ \ \ \ \ \ \ \ \ \ \ \ \ \ } к.ф.-м.н., доцент Иванков А.А.
\vspace*{0.5cm}

\vspace*{0.3cm}

\begin{flushleft}
Консультанты:\\
\vspace*{0.3cm}
по вопросам информационного поиска \ \ \ \ \ \ \ \underline{ \ \ \ \ \ \ \ \ \ \ \ \ \ \ \ \ \ \ \ \ \ \ \ } к.ф.-м.н. Кураленок И.Е.\\
\vspace*{0.3cm}
по вопросам охраны труда \ \ \ \ \ \ \ \ \ \ \ \ \ \ \ \ \ \ \ \ \underline{ \ \ \ \ \ \ \ \ \ \ \ \ \ \ \ \ \ \ \ \ \ \ \ } к.т.н., доцент Монашков В.В.
\end{flushleft}

\end{center}
\vfill
\begin{center}
{\large Санкт-Петербург \\ 2015}
\end {center}
\end{titlepage}


\topmargin -1cm
\hoffset -0.7in
\textwidth 6.0in
\textheight 9.0in
\parindent 1cm

% 1.5 line spacing
%\renewcommand{\baselinestretch}{1.5}
\setstretch{1.5}
% Use 1.0 spacing for chapter headings
\makeatletter
\patchcmd{\@makechapterhead}{\huge}{\setstretch{1.0}\huge}{}{}
\apptocmd{\@makechapterhead}{}{}{}
\makeatother

\pagenumbering{arabic}
\setcounter{tocdepth}{4}
\normalsize

\renewcommand{\contentsname}{Содержание}
\tableofcontents

\chapter*{Введение}
\addcontentsline{toc}{chapter}{Введение}

% План:
% Мотивация
% - о важности информационного поиска и поисковых систем
% - о важности задачи ранжирования
% - о развитии поисковых систем (тренд -- внедрение результатов вертикальных поисков) + специализированные результаты для сценариев
% - о важности задачи ранжирования разнородных результатов


\chapter{Обзор литературы}



\chapter{Постановка задачи}

\chapter*{Заключение}
\addcontentsline{toc}{chapter}{Заключение}

В данной работе предложен новый метод ранжирования разнородных результатов поиска. Главная отличительная особенность метода состоит в том, что результаты поиска располагаются исходя из соображений максимизации релевантности всей поисковой выдачи в целом, а не в соответствии с релевантностями отдельных результатов. Благодаря этому метод является универсальным -- он может быть применен к разнообразным видам поисковых результатов и для разных моделей поисковой выдачи. Также переход от рассмотрения поисковых результатов по отдельности к рассмотрению выдачи в целом позволяет естественным образом учитывать зависимости и отношения между разными типами результатов. \note{Еще преимущества?}

% Какие еще преимущества?
% - легкое добавление нового типа результата / результатов от нового источника ?
%

% Про события, которые считаем успешными?

Предложенный метод был реализован и применен для встраивания 31 типа специализированных результатов вертикальных поисковых источников в мобильную поисковую выдачу системы Яндекс. Использовались специализированные результаты поиска по картинкам, видео, мобильным приложениям, поиска товаров, новостей, погоды, результаты гео-поиска и других сервисов компании Яндекс. Была проведена оценка качества работы метода [ref] и сравнение с текущим используемым методом встраивания специализированных результатов [ref] по метрикам, основанным на асессорских оценках:  по точности и полноте показа специализированных результатов и метрике \textit{pfound} [ref] \note{+ online-метрики?}. Сравнение показало улучшение точности показа специализированных результатов на 21.22\% при снижении полноты на 29.21\% и прирост качества по метрике \textit{pfound} на 0.27\%. \todonote{уточнить результаты}

В ходе реализации метода и встраивания его в поисковую систему также была решена задача эффективного нахождения аргумента максимизации функции, представляющей собой ансамбль решающих деревьев специального вида (oblivious decision trees), и нахождения заданного числа кандидатов в аргументы максимизации при наличии частично вычисленного вектора признаков [ref]. Решение этой задачи позволяет избежать задания запросов к тем поисковым источникам, результаты которых будут заведомо нерелевантны заданному поисковому запросу. Также следует отметить, что решение данной задачи имеет самостоятельную ценность, и может быть применено не только для реализации предложенного метода ранжирования разнородных результатов поиска, но и в других задачах.

\end{document}


